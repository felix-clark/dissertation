\chapter{Experimental apparatuses}

\section{The Large Hadron Collider}
\subsection{Accelerator physics}
\subsection{Injection chain}
\subsection{LHC Experiments}
The two largest LHC experiments, ATLAS (A Toroidal LHC Apparatus) \cite{Aad:2008zzm} and CMS (Compact Muon Solenoid) \cite{Chatrchyan:2008aa}, are general-purpose detectors that fulfilled one of their primary objectives with the joint discovery of the Higgs boson in 2012 \cite{Aad:2012tfa,Chatrchyan:2012xdj}. They continue to be used in the search for physics beyond the Standard Model such as supersymmetry or large extra dimensions. The large rapidity coverage of their calorimeter and tracking systems also make them highly capable of measuring both low- and high-energy probes of heavy ion collisions. The ATLAS detector, which provided the data used in this thesis, will be discussed in more detail in \Sect{\ref{sec:atlas}}.

ALICE (A Large Ion Collider Experiment) \cite{Aamodt:2008zz} is a dedicated heavy-ion detector that is particularly adept at the identification of particle species.

With many subdetectors spread over a forward region from its interaction point, the LHCb (Large Hadron Collider Beauty) experiment \cite{Alves:2008zz} can make detailed measurements of the decay products of b-quarks. Recently it has led to the observation of possible pentaquark states \cite{Aaij:2015tga}.

TOTEM (Total Cross Section, Elastic Scattering and Diffraction Dissociation Measurement at the LHC) \cite{Anelli:2008zza} has detectors in the forward region over 200 m on either side of the CMS interaction point. Its location near the beam pipe puts it in a position to detect the products of elastic and diffractive collisions, which are characterized by a lack of color connection that would otherwise produce particles in the mid-rapidity region.

The LHCf (Large Hadron Collider Forward) experiment \cite{Adriani:2008zz} has two detectors along LHC beamline, at 140 m away from the ATLAS interaction point on either side.

The latest experiment to join the ring is MoEDAL, the Monopole and Exotics Detector at the LHC \cite{Acharya:2014nyr}. It is a mostly passive detector located next to LHCb, with the goal of direct detector of magnetic monopoles or other stable massive particles beyond the Standard Model.

\section{The ATLAS detector}
\label{sec:atlas}

\subsection{Trigger system}
\subsection{Inner detector}
\subsection{Calorimeter system}
\subsection{Muon spectrometer}
This section is maybe not required, since these analyses don't actually use the muon spectrometer.

