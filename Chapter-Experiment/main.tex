\graphicspath{{Chapter-Experiment/figures/}}
\chapter{Experimental design and apparatus}

\section{Scattering experiments}
The first modern scattering experiment could be said to be the Geiger-Marsden experiments in the early 1910s, in which the Rutherford scattering of alpha ($\alpha$) particles by gold foil was observed \cite{Rutherford:1911zz}.
While most $\alpha$ particles pass through the foil with little deflection, a small fraction of them are deflected to extreme angles (\Fig{\ref{fig:rutherford}}).
This result provided evidence that electric charge within the foil was not distributed uniformly, but localized in very small clusters -- the nuclei of the gold atoms.
\begin{figure}
  %% https://www.chegg.com/homework-help/questions-and-answers/rutherford-s-experiment-studied-deflection-scattering-ofalpha-particles-known-helium-nucle-q458296
  \includegraphics[width=0.8\linewidth]{BLB-1070873-Rutherford_v2.jpg}
  \caption{Rutherford scattering of alpha particles by gold foil, demonstrating the existence of the atomic nucleus.}
  \label{fig:rutherford}
\end{figure}

With elementary quantum mechanics, this type of elastic scattering can be understood a little more precisely.
In the Born approximation, which is essentially the weak-potential limit, the quantum amplitude $f$ of a particle with incoming momentum $\mathbf{p}_{i}$ scattering elastically off a target potential $V(\mathbf{x})$ with outgoing momentum $\mathbf{p}_{f}$ is proportional to the Fourier transform of the potential:

\begin{equation}
  f\left(\mathbf{p}_f ; \mathbf{p}_i\right) \propto - \int d^3 x \, V(\mathbf{x}) e^{-i \Delta \mathbf{p} \cdot \mathbf{x}}
  \label{eqn:born}
\end{equation}
where $\Delta \mathbf{p} \equiv \mathbf{p}_f - \mathbf{p}_i$ is the momentum transferred to the projectile particle.
The differential scattering cross section $d\sigma/d\Omega$ (the scattering probability density per unit area) is given by the square of the quantum amplitude, so it is proportional to the squared magnitude of the Fourier transform of the potential.

\begin{equation}
  \frac{d\sigma}{d\Omega} \propto \left| \int d^3 x \, V(\mathbf{x}) e^{-i \Delta \mathbf{p} \cdot \mathbf{x} / \hbar} \right|^2
\end{equation}
Because the net momentum transfer is constrained by the relative momentum between projectile and target, larger momentum (or equivalently, energy) is required to probe finer structure of the target.

\subsection{Accelerator physics}


\section{The Large Hadron Collider}
\subsection{Injection chain}
\subsection{LHC Experiments}
The two largest LHC experiments, ATLAS (A Toroidal LHC Apparatus) \cite{Aad:2008zzm} and CMS (Compact Muon Solenoid) \cite{Chatrchyan:2008aa}, are general-purpose detectors that fulfilled one of their primary objectives with the joint discovery of the Higgs boson in 2012 \cite{Aad:2012tfa,Chatrchyan:2012xdj}. They continue to be used in the search for physics beyond the Standard Model such as supersymmetry or large extra dimensions. The large rapidity coverage of their calorimeter and tracking systems also make them highly capable of measuring both low- and high-energy probes of heavy ion collisions. The ATLAS detector, which provided the data used in this thesis, will be discussed in more detail in \Sect{\ref{sec:atlas}}.

ALICE (A Large Ion Collider Experiment) \cite{Aamodt:2008zz} is a dedicated heavy-ion detector that is particularly adept at the identification of particle species.

With many subdetectors spread over a forward region from its interaction point, the LHCb (Large Hadron Collider Beauty) experiment \cite{Alves:2008zz} can make detailed measurements of the decay products of b-quarks. Recently it has led to the observation of possible pentaquark states \cite{Aaij:2015tga}.

TOTEM (Total Cross Section, Elastic Scattering and Diffraction Dissociation Measurement at the LHC) \cite{Anelli:2008zza} has detectors in the forward region over 200 m on either side of the CMS interaction point. Its location near the beam pipe puts it in a position to detect the products of elastic and diffractive collisions, which are characterized by a lack of color connection that would otherwise produce particles in the mid-rapidity region.

The LHCf (Large Hadron Collider Forward) experiment \cite{Adriani:2008zz} has two detectors along LHC beamline, at 140 m away from the ATLAS interaction point on either side.

The latest experiment to join the ring is MoEDAL, the Monopole and Exotics Detector at the LHC \cite{Acharya:2014nyr}. It is a mostly passive detector located next to LHCb, with the goal of direct detector of magnetic monopoles or other stable massive particles beyond the Standard Model.

\section{The ATLAS detector}
\label{sec:atlas}

\subsection{Trigger system}
\subsection{Inner detector}
\subsection{Calorimeter system}
\subsection{Muon spectrometer}
This section is maybe not required, since these analyses don't actually use the muon spectrometer.

