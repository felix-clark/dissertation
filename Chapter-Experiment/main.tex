\graphicspath{{Chapter-Experiment/figures/}}
\chapter{Experimental design and apparatus}

\section{Scattering experiments}
The first modern scattering experiment could be said to be the Geiger-Marsden experiments in the early 1910s, in which the Rutherford scattering of alpha ($\alpha$) particles by gold foil was observed \cite{Rutherford:1911zz}.
While most $\alpha$ particles pass through the foil with little deflection, a small fraction of them are deflected to extreme angles (\Fig{\ref{fig:rutherford}}).
This result provided evidence that electric charge within the foil was not distributed uniformly, but localized in very small clusters -- the nuclei of the gold atoms.
\begin{figure}[ht]
  %% https://www.chegg.com/homework-help/questions-and-answers/rutherford-s-experiment-studied-deflection-scattering-ofalpha-particles-known-helium-nucle-q458296
  \includegraphics[width=0.8\linewidth]{BLB-1070873-Rutherford_v2.jpg}
  \caption{Rutherford scattering of alpha particles by gold foil, demonstrating the existence of the atomic nucleus.}
  \label{fig:rutherford}
\end{figure}

With elementary quantum mechanics, this type of elastic scattering can be understood a little more precisely.
In the Born approximation, which is essentially the weak-potential limit, the quantum amplitude $f$ of a particle with incoming momentum $\mathbf{p}_{i}$ scattering elastically off a target potential $V(\mathbf{x})$ with outgoing momentum $\mathbf{p}_{f}$ is proportional to the Fourier transform of the potential:

\begin{equation}
  f\left(\mathbf{p}_f ; \mathbf{p}_i\right) \propto - \int d^3 x \, V(\mathbf{x}) e^{-i \Delta \mathbf{p} \cdot \mathbf{x}}
  \label{eqn:born}
\end{equation}
where $\Delta \mathbf{p} \equiv \mathbf{p}_f - \mathbf{p}_i$ is the momentum transferred to the projectile particle.
The differential scattering cross section $d\sigma/d\Omega$ (the scattering probability density per unit area) is given by the square of the quantum amplitude, so it is proportional to the squared magnitude of the Fourier transform of the potential.

\begin{equation}
  \frac{d\sigma}{d\Omega} \propto \left| \int d^3 x \, V(\mathbf{x}) e^{-i \Delta \mathbf{p} \cdot \mathbf{x} / \hbar} \right|^2 = \left| \tilde{V}(\mathbf{\Delta \mathbf{p}}) \right|^2
\end{equation}
Because the net momentum transfer is constrained by the relative momentum between projectile and target, larger momentum -- or equivalently, higher energy -- is required to probe finer structure of the target.
This correspondence between center-of-mass energy and the spatial resolution of the probe remains valid even for inelastic collisions and strong interactions.
The required energy to probe a distance scale can be estimated with dimensional analysis.
To resolve the structure of a target down to a distance scale $l$, the center-of-mass energy $E$ is

\begin{equation}
E = \frac{\hbar c}{l} \approx \frac{0.2 \GeV \fm}{l} \; .
\end{equation}
Resolving individual atoms requires collisions with energy of order \keV, resolving the nucleus requires at least \MeV, and resolving sub-nucleic structure requires collisions of at least \GeV energies.

\subsection{Accelerator physics}

In a fixed-target experiment using target particles of mass $m_t$ at rest and a projectile beam with particles of mass $m_p$ and energy $E$, the center-of-momentum energy $\sqrt{s}$ is given by
\begin{equation}
\sqrt{s} = \sqrt{(m_t+m_p)^2 + 2 m_t (E - m_p)} \; ,
\end{equation}
which scales with the square root of the beam energy even for large energies.
On the other hand, if two beams are accelerated and directed into each other, the center-of-momentum energy of each collision is
\begin{equation}
\sqrt{s} = \sqrt{(E_t + E_p)^2 - (|\mathbf{p}_t| - |\mathbf{p}_p|)^2} \; ,
\end{equation}
which scales linearly with the total energy, and is equal to $2E$ in the case where the beams are identical.
While fixed-target experiments were the first to be developed, they are not as efficient at reaching high energies as dual-beam colliders.

\begin{figure}[t]
  %%  https://www.cyberphysics.co.uk/topics/atomic/Accelerators/LINAC/Linac.htm
  \includegraphics[width=0.8\linewidth]{LINAC.png}
  \caption{A linear accelerator.}
  \label{fig:linac}
\end{figure}

The most straightforward method to accelerate charged particles to high energies is to allow them to pass through a large voltage differential.
A large voltage can be produced, for instance, with a Van de Graaff generator \cite{PhysRev.43.149}.
The voltage differential is limited by the insulation breakdown, which in practice caps the energy of accelerated particles with charge $Ze$ to a few $Z \cdot\MeV$.
A modern linear particle accelerator (linac) circumvents this issue by passing ions through a series of drift tubes with alternating positive and negative potentials (\Fig{\ref{fig:linac}}).
The drift tubes are constructed with conducting material, so they shield the ions traveling through them from external electric acceleration.
Between the drift tubes, however, strong electric fields are induced by the alternating potentials.
This voltage difference is oscillated with radio frequency and the apparatus is constructed such that ions injected at a given velocity can pass through each gap with acceleration in the forward direction.
The Stanford Linear Accelerator Center (SLAC) hosts the largest linear accelerator, which began operation in 1966.
The 3 km long machine was capable of accelerating electrons and positrons to energies of up to 50 \GeV.
While this is several orders of magnitude higher than the energies accessible with simple voltage differential, increasing the energy further requires proportional increases to the length of the accelerator, exacerbating technical difficulties in its placement, construction, and maintenance.

\begin{figure}[t]
 %% https://www.mpoweruk.com/figs/cyclotron.htm
  \includegraphics[width=0.8\linewidth]{cyclotron.png}
  \caption{A cyclotron.}
  \label{fig:cyclotron}
\end{figure}
The cyclotron is the next fundamental progression in accelerator technology.
Ions in a cyclotron are passed between two semi-cylindrical electrodes (``dees'') that are driven with an alternating voltage (\Fig{\ref{fig:cyclotron}}).
A large electromagnet keeps the particles traveling in a circular path contained within the dees.
At non-relativistic energies, the velocity of the ions is proportional to the radius of their path, so the time taken for one revolution is independent of the velocity.
If the voltage between the electrodes is oscillated with a frequency equal to the cyclotron resonance frequency $f_\textrm{cyclo} = q B / 2\pi m$, then ions are accelerated to higher energies with each pass between the dees.
At a fixed radius (and therefore fixed energy), the ion beam is allowed to exit the cyclotron.
Classically, the kinetic energy is proportional to the square of the radius, inviting a comparison to a hypothetically coiled linac.
Relativistic energies can be attained with more sophisticated designs that vary either the frequency over time or the magnetic field with the radial position, but as the velocity approaches $c$ the energy is only linearly proportional to the radius:
\begin{align}
\label{eqn:gyroradius}
E =& \; \frac{q B R c^2}{v}\\
\overrightarrow{{}_{v \rightarrow c}}& \; q B R c
\end{align}
This shows that the energies accessible from a circular accelerator are directly limited by the maximum strength of the magnetic field and the radius of the path.

The radius of a cyclotron design can only be increased so much before it becomes infeasible.
The synchrotron is a toroidal accelerator through which ions travel in a set circular path.
This layout does not permit it to accelerate particles from an arbitrarily small energy, so a synchrotron is generally filled with a linac first, then the energy of the ions in its ring are increased.
The limiting factor for synchrotron performance depends on the mass of the particle; for electrons the power lost via synchrotron radiation $P_\textrm{synch-rad} \propto e^2 E^4 / m^4 R^2 $ is the limiting factor in the maximum energy.


\section{The Large Hadron Collider}

The Large Hadron Collider (LHC) is a high-energy particle ring collider located near Geneva on the Swiss-French border \cite{LHCMachine}.
It was constructed by the European Organization for Nuclear Research (CERN, for ``Conseil européen pour la recherche nucléaire'') and is currently the highest energy particle accelerator in the world.
While it was primarily designed to provide proton-proton (\pp) collisions, it is also capable of colliding lead (${}^{208}_{\ 82}\textrm{Pb}$) and xenon (${}^{129}_{\ 54}\textrm{Xe}$) ions with themselves and with protons.
The results of this thesis will use data collected from the 2013 \pPb collisions, which were taken at a center-of-mass energy per nucleon of \pPbenergy.

\subsection{Injection chain}

\begin{figure}[ht]
\includegraphics[width=0.8\linewidth]{LHCinjectionchain.jpg}
\caption{The LHC is the last ring (dark blue line) in a complex chain of particle accelerators. The smaller machines are used in a chain to help boost the particles to their final energies and provide beams to a whole set of smaller experiments. (Copyright CERN \cite{Mobs:2197559})}
\label{fig:injection_chain}
\end{figure}

Before being injected into the LHC, ion beams pass through a series of increasingly large accelerators, incrementally raising their energy.
The CERN accelerator complex is sketched in \Fig{\ref{fig:injection_chain}}.
%% compare different soruces for protons and lead.

\subsection{LHC experiments}
The two largest LHC experiments, ATLAS (A Toroidal LHC Apparatus) \cite{Aad:2008zzm} and CMS (Compact Muon Solenoid) \cite{Chatrchyan:2008aa}, are general-purpose detectors that fulfilled one of their primary objectives with the joint discovery of the Higgs boson in 2012 \cite{Aad:2012tfa,Chatrchyan:2012xdj}.
They continue to be used in the search for physics beyond the Standard Model such as supersymmetry and large extra dimensions.
The large rapidity coverage of their calorimeter and tracking systems also make them highly capable of measuring both low- and high-energy probes of heavy ion collisions.
The ATLAS detector, which provided the data used in this thesis, is discussed in more detail in \Sect{\ref{sec:atlas}}.

ALICE (A Large Ion Collider Experiment) \cite{Aamodt:2008zz} is a dedicated heavy-ion detector that is particularly adept at the identification of particle species.

With many subdetectors spread over a forward region from its interaction point, the LHCb (Large Hadron Collider Beauty) experiment \cite{Alves:2008zz} can make detailed measurements of the decay products of b-quarks. Recently it has led to the observation of possible pentaquark states \cite{Aaij:2015tga}.

TOTEM (Total Cross Section, Elastic Scattering and Diffraction Dissociation Measurement at the LHC) \cite{Anelli:2008zza} has detectors in the forward region over 200 m on either side of the CMS interaction point.
Its location near the beam pipe puts it in a position to detect the products of elastic and diffractive collisions, which are characterized by a lack of color connection that would otherwise produce particles in the mid-rapidity region.

The LHCf (Large Hadron Collider Forward) experiment \cite{Adriani:2008zz} has two detectors along LHC beamline, at 140 m away from the ATLAS interaction point on either side.

The latest experiment to join the ring is MoEDAL, the Monopole and Exotics Detector at the LHC \cite{Acharya:2014nyr}.
It is a mostly passive detector located next to LHCb, with the goal of direct detector of magnetic monopoles or other stable massive particles beyond the Standard Model.

\section{The ATLAS detector}
\label{sec:atlas}

\subsection{Trigger system}
\subsection{Inner detector}
\subsection{Calorimeter system}
\subsection{Muon spectrometer}
This section is maybe not required, since these analyses don't actually use the muon spectrometer.

