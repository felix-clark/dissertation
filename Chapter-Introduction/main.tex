\chapter{Introduction}
\label{ch:intro}

%% quotes to potentially sprinkle:
%% intro: ``In the beginning the Universe was created. This has made a lot of people very angry and been widely regarded as a bad move.'' -- Douglas Adams
%% acknowledgements: ``Getting an education was a bit like a communicable sexual disease. It made you unsuitable for a lot of jobs and then you had the urge to pass it on.'' -- Terry Pratchett
%% theory lol: ``I can believe things that are true and things that aren't true and I can believe things where nobody knows if they're true or not.'' -- Neil Gaiman
%% or: ``Maturity, one discovers, has everything to do with the acceptance of 'not knowing'.'' -- Mark Z. Danielewski
%% results: ``The most exciting phrase to hear in science, the one that heralds new discoveries, is not 'Eureka!', but 'That's funny...'' --- Isaac Asimov

 notes:

 we see collectivity in even fairly low multiplicity pp, and it doesn't look particularly like hydro (?) (see Z-tagged pp v2). refer to AMPT (, CGC ?) models that predict v2 in pp. this thesis will argue that the collectivity in central \pPb has specific features that suggest hydro -- it's not clear in low multiplicity.

 
why HI physics? %% \cite{Busza:2018rrf} for big picture, but no citations in intro?
cosmology, very early universe. HI collisions are ``little bangs''
phase diagram of \qcd -- essentially phase diagram of \emph{matter}
how does strongly-coupled fluid emerge from constituents that are weakly-coupled at short length scales?

Matter at extremely high temperatures and densities is studied with relativistic \ac{HI} collisions at \ac{RHIC} and the \ac{LHC}.
When nuclei with large atomic number ($Z \gg 1$) collide at velocities approaching the speed of light $c$, the constituent nucleons melt into a hot and dense fluid.
These ``little bangs'' generate the state of matter present in the first microsecond of the universe.


%% \cite{HION-2015-11} %% our paper

This dissertation is organized as follows.
\Cref{ch:background} discusses the historical background of nuclear physics and some of the developments made in the study of heavy ion collisions.
It also introduces the framework for understanding femtoscopic measurements in general.
\Cref{ch:experiment} motivates the some of the design for high-energy collider experiments and describes several components of the \ac{LHC} and the ATLAS detector.
The process of reconstructing charged particles and identifying pions is discussed in \cref{ch:reconstruction}.
\Cref{ch:analysis} provides the details of the measurement process for the analyses and enumerates the systematic effects and uncertainties.
The results of the ATLAS femtoscopy measurements in \pPb collisions are presented with discussion in \cref{ch:results}, including some comparisons to theoretical hydrodynamic models.
