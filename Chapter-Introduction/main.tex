\chapter{Introduction}
\label{ch:intro}

%% quotes to potentially sprinkle:
%% intro: ``In the beginning the Universe was created. This has made a lot of people very angry and been widely regarded as a bad move.'' -- Douglas Adams
%% acknowledgements: ``Getting an education was a bit like a communicable sexual disease. It made you unsuitable for a lot of jobs and then you had the urge to pass it on.'' -- Terry Pratchett
%% theory lol: ``I can believe things that are true and things that aren't true and I can believe things where nobody knows if they're true or not.'' -- Neil Gaiman
%% or: ``Maturity, one discovers, has everything to do with the acceptance of 'not knowing'.'' -- Mark Z. Danielewski
%% results: ``The most exciting phrase to hear in science, the one that heralds new discoveries, is not 'Eureka!', but 'That's funny...'' --- Isaac Asimov

 
%% why HI physics? %%  for big picture, but no citations in intro?
%% cosmology, very early universe. HI collisions are ``little bangs''
%% phase diagram of \qcd -- essentially phase diagram of \emph{matter}
%% how does strongly-coupled fluid emerge from constituents that are weakly-coupled at short length scales?

%% CMS summary in pp, pPb, PbPb from Dec 2017: \cite{Sirunyan:2017ies}

Matter at extremely high temperatures and densities is studied with relativistic \ac{HI} collisions at the \ac{RHIC} and the \lhc.
When nuclei with large atomic number ($A \gg 1$) collide at velocities approaching the speed of light $c$, the constituent nucleons melt into a hot and dense fluid.
The ``little bangs'' in these \AA collisions produce the state of matter present in the first microsecond of the universe.
Studying the properties of these collisions constrains the phase diagram of matter at high temperatures and densities and provides another avenue for understanding the evolution of the early universe \cite{Busza:2018rrf}.
%% The bulk of the matter behaves like a strongly-coupled perfect fluid with a near-minimal specific viscosity $\eta/s$.

Nuclear matter is now understood to be described by \qcd, a non-abelian gauge \ac{QFT} defined by a simple symmetry but with rich and deep phenomenology.
The study of \qcd matter is really the study of matter, as most of the mass in the universe comes from the energy of the quarks and gluons (partons) bound within protons and neutrons (nucleons).
It is not obvious from first principles that \qcd at high temperatures exhibits a deconfined state of matter, in which the partons are not bound but free to travel within the medium; however, state-of-the-art numerical computations that run \qcd on a discrete lattice show a crossover phase transition.
A deconfined phase of strongly-coupled quarks and gluons, the \qgp, could be expected to behave like a fluid.
The defining feature of hydrodynamics is the complementary relationship between spatial anisotropies and momentum anisotropies as hydrodynamics predicts a fluid acceleration in the direction of the negative gradient of the fluid pressure \cite{Kolb:2003dz}. %% $-\nabla p$.
The experimentally measured Fourier components of the particle production in \AA collisions are consistent with hydrodynamic evolution of a fluid with a near-minimal specific viscosity $\eta/s$.
By contrast, a weakly-coupled gas would not have significant momentum anisotropies as the partons would easily pass through the medium.

Sprays of particles produced in high energy collisions, known as jets, are suppressed in \AA collisions.
This indicates that the jets are being quenched by a medium that forms before the jets can escape.
The transverse momentum of dijets\footnote{pairs of jets that are back-to-back in the transverse plane} produced in \AA collisions have a greater asymmetry than those in \pp collisions, indicating that they dissipate different amounts of energy as they traverse different path lengths in a medium \cite{HION-2010-02,HION-2012-11}.
This dijet asymmetry is not observed in \pA collisions where the initial transverse size of the collision is constrained to about 2 \fm across.
If a hydrodynamic fluid is produced in such collisions it is likely to have a formation time on the order of $1~\fm/c$, by which time any jets have traveled outside of the collision region.

It is not clear that hydrodynamics should apply in \pA and \pp collisions where the system size may not be significantly larger than the mean free path of the constituents.
However, the Fourier components of particle production in small systems are suggestive of flow-like behavior, and even low-multiplicity \pp collisions have a significant 2nd-order Fourier coefficient $v_2$.
There are alternative explanations for this phenomenon like the \ac{CGC} description \cite{Iancu:2000hn} or \ac{AMPT} model \cite{Lin:2004en}, and some recent measurements suggest that the origin of the $v_2$ in \pp collisions may not be hydrodynamical \cite{ATLAS-CONF-2017-068}.
A contemporary focus of high energy nuclear physics is determining the domain of validity for a hydrodynamic description of \qcd matter.

The technique of \emph{femtoscopy} can address this question by providing information about the spatio-temporal evolution of the particle sources produced in nuclear collisions \cite{Lisa:2005dd}.
Femtoscopic measurements in \AuAu collisions at \ac{RHIC} and \PbPb collisions at the \lhc have been consistent with a short-lived hydrodynamically expanding source.
Comparable analyses in \pPb show similar signs of expansion in central collisions with a large number of nucleon participants \Npart, but the situation is not as clear in peripheral collisions \cite{Abelev:2014pja,Adam:2015pya}.
In particular, the source is observed to be contracted along the axis of greatest particle flow in \AuAu \cite{Adams:2003ra,Adare:2014vax,Adamczyk:2014mxp} and \PbPb \cite{Adamova:2017opl} collisions.
This direct correspondence between initial spatial anisotropy and final momentum anisotropy is a prominent signal for hydrodynamics.

This thesis presents femtoscopic measurements of the source size and shape in \pPb collisions with the ATLAS detector from \Ref{\cite{HION-2015-11}}, which provides a finer level of detail than previous measurements.
Azimuthally-dependent results in central \pPb collisions are also presented, parts of which have been published in \Ref{\cite{ATLAS-CONF-2017-008}}.
Both of these analyses introduce technical improvements to correction procedures compared to what has been used so far in the literature.
The results show clear evidence that the collective behavior in central \pPb is hydrodynamic in nature, but these signatures are reduced in peripheral collisions.

This dissertation is organized as follows.
\Cref{ch:background} discusses the historical background of nuclear physics and some of the developments made in the study of heavy ion collisions.
It also introduces the framework for understanding femtoscopic measurements in general.
\Cref{ch:experiment} motivates the design for high-energy collider experiments and describes several components of the \ac{LHC} and the ATLAS detector.
The process of reconstructing charged particles and identifying pions is discussed in \cref{ch:reconstruction}.
\Cref{ch:analysis} provides the details of the measurement process for the analyses and enumerates the systematic effects and uncertainties.
The results of the ATLAS femtoscopy measurements in \pPb collisions are presented with discussion in \cref{ch:results}, including some comparisons to theoretical hydrodynamic models.

\subsection*{Conventions and definitions}
In most cases dimensionless units are used such that $c = \hbar = k_B = 1$, although occasionally the constants will be left in an equation that expresses a specific physical scale. The time-positive Minkowski metric is used in relativistic expressions, so that $g_{\mu\nu} = \mathrm{diag}\left( 1, -1, -1, -1\right)$.

The ATLAS coordinate system is used, which is right-handed with its origin at the nominal \ac{IP} in the center of the detector and the $z$-axis along the beam pipe.
%% The $x$-axis points from the \ac{IP} to the center of the LHC ring, and the $y$-axis points upward.
Cylindrical coordinates $(r,\phi)$ are used in the transverse plane, where $\phi$ is the azimuthal angle around the beam pipe.
Transverse momentum is denoted by \pt.
The pseudorapidity is defined in terms of the polar angle from the beam line $\theta$ as $\eta=-\ln\tan(\theta/2)$ and represents the massless limit of the rapidity $y = \frac{1}{2} \ln \frac{E + p_z}{E - p_z}$.
Both rapidity and pseudorapidity transform under a longitudinal boost with relative velocity $v_z$ by an additive constant of $\tanh^{-1} v_z$, so differences of these quantities are boost-invariant.
