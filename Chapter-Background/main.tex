\chapter{Theoretical and phenomenological background}
\label{ch:background}
\graphicspath{{Chapter-Background/figures/}}

\section{Quantum chromodynamics}
\subsection{History and experimental motivation}

The atomic nucleus was discovered in the early 20th century \cite{Rutherford:1911zz}, and a few years later it was determined that it was composed of protons ($p$).
An additional, electrically-neutral nuclear constituent particle was proposed soon after and the neutron ($n$) was finally discovered in the early 1930s \cite{Chadwick:1932ma}.
Clearly some new type of interaction, strong enough to overcome electrostatic repulsion between protons, bound the nucleons ($p$ and $n$) together in the nucleus.
A particle field was proposed to mediate this strong interaction, dubbed the pion ($\pi^\pm$, $\pi^0$) \cite{Yukawa:1935xg}.
Since the strong interaction only acts over a length scale of about 2 fm, the mediating pion was predicted to have a mass of about 100 \MeV\footnote{The potential mediated by a boson of mass $m$ is proportional to \( - \exp(-mcr/\hbar)/r\), where $r$ is the separation between a pair of participants. Thus the mass of a mediating boson for a potential with a characteristic cutoff length of $\lambda$ is $m = \hbar/c\lambda = \frac{200 \MeVcc \textrm{fm}}{\lambda}$.}.
The charged pion was indeed discovered experimentally in 1947 \cite{Lattes:1947mw}.
Pions can be interpreted as the Nambu-Goldstone bosons corresponding to the spontaneously broken chiral symmetry, which explains their relatively small masses.

The expanding body of observed hadrons\footnote{a particle bound by the strong nuclear interaction} and their allowed decays suggested additional conserved quantum numbers suggested that they could be composed of constituent particles.
It was noticed that hadrons can be organized by their quantum numbers in a manner described by an $SU(3)$ \emph{flavor} symmetry \cite{GellMann:1962xb}.
This suggests that hadrons are composed of constituent particles called \emph{quarks}; however, the quark model was not immediately accepted because free quarks were not (and still have not been) observed.
Because electrons are point-like particles to the limits of current experiments, a scattering process $e^- A \rightarrow e^- X$ is dependent only on the internal structure of the target $A$.
The electron scatters through a virtual photon that interacts directly with the hadron in a process called \ac{DIS}\footnote{``deep'' because it probes the constituent structure of the target, ``inelastic'' because kinetic energy and matter are exchanged in the creation of the product particles}.

\begin{figure}[t]
  \includegraphics{dis_electron_proton.png}
  \caption{Deep inelastic scattering of a lepton on a hadron. Figure from \Ref{\cite{dis_fig_proceedings}}.
}
  \label{fig:dis}
\end{figure}

Under the constraints of Lorentz and gauge invariance, the cross-section of an unpolarized \ac{DIS} process with incoming lepton and proton momenta $k$ and $P$ respectively and momentum transfer $q$ can be expressed as \cite{Tanabashi:2018oca}
\begin{equation}
  \frac{d^2 \sigma}{dx \, dQ^2} = \frac{4\pi\alpha^2_\textrm{EM}}{2xQ^4}\left[ \left(1+(1-y)^2\right) F_2\left(x, Q^2\right) - y^2 F_L \left(x, Q^2\right) \right]
  \label{eq:dis}
\end{equation}
where $Q^2 = -q^2$ is the absolute magnitude squared of the photon's virtuality, $x = \frac{Q^2}{2q \cdot P}$, and $y = \frac{q \cdot P}{k \cdot P}$ is the fraction of the lepton's energy lost in the nucleon's rest frame.
In the parton model, which describes the proton as approximately-free point-like quarks in the infinite longitudinal momentum frame, $x$ is interpreted as the fraction of the target proton's momentum carried by a struck parton.
The structure functions $F_i(x, Q^2)$ describe the inherent internal structure of the proton.
The longitudinal structure function $F_L = F_2 - 2xF_1$ is zero by the Callan-Gross relation \cite{Callan:1969uq}, and experimentally the ratio $2xF_1 / F_2$ is consistent with 1 independent of $x$.
In the parton model, where the proton is described in terms of free point-like quarks constituents, the structure function $F_2$ is decomposed into \acp{PDF}
\begin{equation}
F_2 \left(x, Q^2\right) = x \sum_q e_q^2 f_{q/p}(x)
\end{equation}
to lowest order in the strong coupling constant $\alpha_s$.
The independence of $Q^2$ of the \acp{PDF} is a manifestation of their point-like description in the parton model and is known as \emph{Bjorken scaling}.
Logarithmic corrections are understood theoretically and come from as gluon radiation from the quarks becomes relevant, particularly at small $x$ (\Cref{fig:proton_f2}).

\begin{figure}[t]
  %% page 326 of particle review
  \includegraphics[width=0.45\linewidth]{proton_f2.png}
  \includegraphics[width=0.54\linewidth]{proton_f2_vs_x.png}
  \caption{The structure function $F_2\left(x, Q^2\right)$ of the proton as a function of $Q^2$ (left) and of $x$ (right). Figure from \Ref{\cite{Tanabashi:2018oca}}.}
  \label{fig:proton_f2}
\end{figure}

%% TODO: could discuss R_had as evidence for spin-1/2 quarks ??

An experimental and phenomenological description of the nucleon structure functions does not provide a complete description of the interactions among nucleon constituents.
The success of gauge theories in describing \ac{QED} ($U(1)$) and electroweak theory ($U(2) = SU(2) \otimes U(1)$) suggests that some other gauge theory may be able to describe the nuclear interaction.
A quantum field theory with an $SU(N_c)$ ``color'' symmetry with $N_c$ colors, called \qcd, is one such candidate.
For it to be a believable description of the strong interaction, it must not only be consistent with experimental observations, but it must also explain why free quarks have never been observed.
The rate of hadronic production in electron-positron collisions is proportional to $N_c$, so experimental measurements of the ratio\footnote{at energies above the $b\bar{b}$ threshold and below the mass of the $Z$ boson}
\begin{equation}
R \equiv \frac{\sigma\left(e^+ e^- \rightarrow \textrm{hadrons}\right)}{\sigma\left(e^+ e^- \rightarrow \mu^+ \mu^- \right)} \approx N_c \sum_{q \in \{u,d,s,c,b\}} e_q^2 = \frac{11}{9} N_c
\end{equation}
have been made to determine the number of colors, showing very good agreement with a value of $N_c = 3$.
Though the non-abelian character of \qcd makes many practical calculations difficult, it has shown remarkable success in describing the strong interaction, as will be discussed in the remainder of this section.

\subsection{The QCD Lagrangian}
The gauge-invariant lagrangian density of \qcd \cite{Wilczek:2000ih} is given by
\begin{equation}
  \Lagr_\mathrm{QCD} \equiv -\frac{1}{4} G^a_{\mu\nu}G^{a\mu\nu} + \bar{\psi} \left( i \slashed{D} - m \right) \psi \; .
\end{equation}
Here repeated indices are summed, where $\mu$ and $\nu$ indicate spacetime indices, $a$, $b$, and $c$ indicate color indices in the fundamental ($N=3$) representation, and $A$, $B$, and $C$ indicated color indices in the adjoint ($N=8$) representation.
The slash notation refers to contraction with the gamma matrices $\{\gamma^\mu, \gamma^\nu\} = 2\eta^{\mu\nu}$\footnote{with Minkowski signature $(+---)$}.
The covariant derivative
\[ D_\mu \equiv \partial_\mu - i g A^C_\mu t^C\]
where $A^C_\mu$ is the gluon field, $t^C$ are the generators of the $SU(3)$ gauge group, and $g$ is the strong charge constant.
The constant $g$ in the lagrangian is always squared when computing rates from quantum amplitudes, so physical results are typically expressed in terms of
\begin{equation}
  \alpha_s \equiv \frac{g^2}{4\pi} \; ,
\end{equation}
typically called the strong coupling constant.
The gluon field strength tensor is
\[ G^A_{\mu\nu} \equiv \partial_\mu A^A_\nu - \partial_\nu A^A_\mu + g f^{ABC} A^B_\mu A^C_\nu \]
with the $SU(3)$ structure constants defined such that
\[ [t^A,t^B] = if^{ABC}t^C \; \footnote{In the $SU(2)$ gauge group the adjoint representation is three-dimensional and the structure constants are given by the anti-symmetric Levi-Civita symbol $\epsilon^{ABC}$}.\]
The quark fields are defined such that the mass matrix $m$ is diagonal:
\[ \bar{\psi}m\psi = \sum_{q = u,d,s,\ldots} m_{q}\bar{q}q \]
The quark masses $m_q$ are generated by the mechanism of spontaneous electro-weak symmetry breaking in which the Higgs field, coupling to fermions and electro-weak gauge bosons, acquires a nonzero vacuum expectation value.
This procedure induces in the \qcd vacuum a breaking of the chiral symmetry $SU(N_f) \times SU(N_f) \rarrow SU(N_f)$ in the massless Lagrangian.

\begin{figure}[t]
  %% page 155 of particle review
  \includegraphics{qcd_feynman.png}
  \caption{Vertices in \qcd along with the corresponding Feynman rules for the amplitude factors. The strong coupling charge is written here as $e_s = \sqrt{4\pi\alpha_s}$, and the gauge parameter is denoted by $\lambda$. Figure from \Ref{\cite{Altarelli:2013tya}}.}
  \label{fig:qcd_feynman}
\end{figure}

%% The QCD Lagrangian has a symmetry under the gauge transformation defined as... %% we don't really need to get explicit about gauge transformations
The physical interaction vertices of \qcd are a 3-point quark-gluon vertex analogous to he \ac{QED} vertex, a 3-gluon vertex and a 4-gluon vertex (\Cref{fig:qcd_feynman}).
A scalar ghost field\footnote{which has a spin of 0 yet anti-commutes like a fermion} also couples to the gluon as is generally necessary in non-abelian gauge theories to prevent over-counting gauge-equivalent states \cite{Faddeev:1967fc}.
The non-abelian nature of \qcd manifests in Feynman diagrams as the gluon self-interaction vertices.
The gluon-gluon interactions make \qcd difficult to calculate with.
In a classical theory, they prevent the principle of superposition from being applied in chromodynamics.
The highly-nontrivial gluon self-interaction is a crucial ingredient in the richness of nuclear physics.


\subsection{Running of the coupling constant} %% maybe doesn't deserve its own section
\begin{figure}[t]
  %% page 155 of particle review
  \includegraphics{running_as.png}
  \caption{Measurements of the strong coupling constant $\alpha_s$ as a function of the energy scale $Q$. Figure from \Ref{\cite{Tanabashi:2018oca}}.}
  \label{fig:running_coupling}
\end{figure}

As is the case in all quantum field theories, the calculation of higher-order (in $\alpha_s$) loop diagram integrals contain divergences.
In many theories, including \qcd \cite{Gross:1973ju}, these infinities can be dealt with with a process called \emph{renormalization}.
One description of renormalization is the introduction of an energy/momentum cutoff scale.
Physical calculations cannot depend on any such scales, so amplitude calculations must be organized such that any dependence on the renormalization scale cancels in a physical result, which can be done order-by-order in perturbation theory.
This process induces a scale-dependence of the coupling constant $\alpha_s$ on the arbitrary renormalization scale $\mu$.
If the value of $\alpha_s$ is known at a particular $\mu = \mu_0$, its dependence on the scale can be determined by calculating the beta-function
\begin{equation}
  \frac{\partial \alpha_s}{\partial \ln \mu^2} \equiv \beta(\alpha_s)
\end{equation}
which to one loop is given by \cite{Gross:1973id}
\begin{equation}
  \beta(\alpha_s) = -\frac{11 N_c - 2 N_f}{12\pi} \alpha_s^2
\end{equation}
for $N_c$ colors and $N_f$ quark flavors.
The solution is
\begin{equation}
  \alpha_s(\mu) = \frac{12\pi}{\left(11N_c - 2N_f\right)\ln\left(\mu^2 / \lqcd^2\right)}
\end{equation}
where \lqcd is defined by a given value for $\alpha_s$ at a scale $\mu_0$
\[
\lqcd = \mu_0 \exp \left( -\frac{6\pi}{\left(11N_c - 2N_f\right)\alpha_s(\mu_0)} \right) \; .
\]
Written this way, the physical meaning of \lqcd becomes apparent: it is the energy scale below which the strong coupling diverges, showing an explicit breakdown of perturbation theory.

\subsection{Color confinement}
\subsection{Asymptotic freedom}
asymptotic freedom explains Bjorken scaling
\section{QCD at high temperatures: the quark-gluon plasma}
\subsection{Experimental evidence for the existence of the QGP}
\subsection{Equation of state from lattice calculations}
\subsection{AdS-CFT correspondence}

\section{Hydrodynamic description of heavy ion collisions}
\subsection{Applicability of hydrodynamics}
low viscosity: large cross-section, low mfp
\subsection{Fourier decomposition of particle production}
\subsection{Success of hydrodynamics}
\subsection{Flow in small systems}

\section{Jet production and fragmentation}
Though the focus of this thesis is not on the physics of jet fragmentation, control of hard processes is essential to correctly extract observable quantities of interest.
This section will describe the physics behind jets.

TODO: Some of this discussion may be better moved elsewhere; for instance, nuclear modification may be appropriate in ``evidence for QGP''
\subsection{Parton model}
lorentz contraction in infinite momentum frame, transverse freeze-out, factorization
\subsection{Fragmentation function}
\subsection{Nuclear modification}


\section{Femtoscopy in heavy ion collisions}
This section will describe from a theoretical perspective what femtoscopy is and what it aims to achieve. The application of these techniques to experimental data will be addressed in \Cref{ch:analysis}.
\subsection{Imaging the source density function}
\subsection{Parameterization of the correlation function}
\subsection{Final-state interactions}
justify ignoring strong force; describe Coulomb correction (save jet fragmentation details for later)
\subsection{Motivation for femtoscopy in proton-lead}
\subsubsection{Collective expansion}

%% HI summary plots: (v2, RAA, etc)
%% https://atlas.web.cern.ch/Atlas/GROUPS/PHYSICS/CombinedSummaryPlots/HION/
