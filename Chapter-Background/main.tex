\chapter{Theoretical and phenomenological background}
\label{ch:background}

\section{Quantum chromo-dynamics}
\subsection{History and experimental motivation}
\subsection{The QCD Lagrangian}
The lagrangian density of QCD is given by
\begin{equation}
  \Lagr_\mathrm{QCD} \equiv -\frac{1}{4} G^a_{\mu\nu}G^{a\mu\nu} + \bar{\psi} \left( i \slashed{D} - m \right) \psi \; .
\end{equation}
Here repeated indices are summed, where $\mu$ and $\nu$ indicate spacetime indices and $a$, $b$, and $c$ indicate color indices.
The covariant derivative
\[ D_\mu \equiv \partial_\mu - i g A^a_\mu t^a\]
where $A^a_\mu$ is the gluon field, $t^a$ are the generators of the $SU(3)$ gauge group, and $g$ is the coupling constant.
The gluon field strength tensor is
\[ G^a_{\mu\nu} \equiv \partial_\mu A^a_\nu - \partial_\nu A^a_\mu + g f^{abc} A^b_\mu A^c_\nu \]
with the $SU(3)$ structure constants defined such that
\[ [t^a,t^b] = if^{abc}t^c \; .\]
The quark fields are defined such that the mass matrix $m$ is diagonal:
\[ \bar{\psi}m\psi = \sum_{q = u,d,s,\ldots} m_{q}\bar{q}q \]
The quark masses $m_q$ are generated by the mechanism of spontaneous electro-weak symmetry breaking in which the Higgs field, coupling to fermions and electro-weak gauge bosons, acquires a nonzero vacuum expectation value.
This procedure induces a breaking of chiral symmetry $SU(N_f) \times SU(N_f) \rarrow SU(N_f)$ in the massless Lagrangian. %% reword. original lagrangian has chiral symmetry, but vacuum does not.

The QCD Lagrangian has a symmetry under the gauge transformation defined as...

\subsection{Running of the coupling constant} %% maybe doesn't deserve its own section
\subsection{Color confinement}
\subsection{Asymptotic freedom}

\section{QCD at high temperatures: the quark-gluon plasma}
\subsection{Experimental evidence for the existence of the QGP}
\subsection{Equation of state from lattice calculations}
\subsection{AdS-CFT correspondence}

\section{Hydrodynamic description of heavy ion collisions}
\subsection{Applicability of hydrodynamics}
low viscosity: large cross-section, low mfp
\subsection{Fourier decomposition of particle production}
\subsection{Success of hydrodynamics}
\subsection{Flow in small systems}

\section{Jet production and fragmentation}
Though the focus of this thesis is not on the physics of jet fragmentation, control of hard processes is essential to correctly extract observable quantities of interest.
This section will describe the physics behind jets.

TODO: Some of this discussion may be better moved elsewhere; for instance, nuclear modification may be appropriate in ``evidence for QGP''
\subsection{Parton model}
lorentz contraction in infinite momentum frame, transverse freeze-out, factorization
\subsection{Fragmentation function}
\subsection{Nuclear modification}


\section{Femtoscopy in heavy ion collisions}
This section will describe from a theoretical perspective what femtoscopy is and what it aims to achieve. The application of these techniques to experimental data will be addressed in \Cref{ch:analysis}.
\subsection{Imaging the source density function}
\subsection{Parameterization of the correlation function}
\subsection{Final-state interactions}
justify ignoring strong force; describe Coulomb correction (save jet fragmentation details for later)
\subsection{Motivation for femtoscopy in proton-lead}
\subsubsection{Collective expansion}

