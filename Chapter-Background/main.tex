\chapter{Theoretical and phenomenological background}
\label{ch:background}

\section{Quantum chromo-dynamics}
\subsection{History and experimental motivation}
\subsection{The QCD Lagrangian}
\subsection{Running of the coupling constant} %% maybe doesn't deserve its own section
\subsection{Asymptotic freedom}
\subsection{Color confinement}

\section{QCD at high temperatures: the quark-gluon plasma}
\subsection{Experimental evidence for the existence of the QGP}
\subsection{Equation of state from lattice calculations}
\subsection{AdS-CFT correspondence}

\section{Hydrodynamic description of heavy ion collisions}
\subsection{Applicability of hydrodynamics}
low viscosity: large cross-section, low mfp
\subsection{Fourier decomposition of particle production}
\subsection{Success of hydrodynamics}
\subsection{Flow in small systems}

\section{Jet production and fragmentation}
While the focus of this thesis is not on the physics of jet fragmentation, control of hard processes is essential to correctly extract observable quantities of interest.
This section will describe the physics behind jets.

TODO: Some of this discussion may be better moved elsewhere; for instance, nuclear modification may be appropriate in ``evidence for QGP''
\subsection{Parton model}
lorentz contraction in infinite momentum frame, transverse freeze-out, factorization
\subsection{Fragmentation function}
\subsection{Nuclear modification}


\section{Femtoscopy in heavy ion collisions}
This section will describe from a theoretical perspective what femtoscopy is and what it aims to achieve. The application of these techniques to experimental data will be addressed in \Cref{ch:analysis}.
\subsection{Imaging the source density function}
\subsection{Parameterization of the correlation function}
\subsection{Final-state interactions}
justify ignoring strong force; describe Coulomb correction (save jet fragmentation details for later)
\subsection{Motivation for femtoscopy in proton-lead}
\subsubsection{Collective expansion}

