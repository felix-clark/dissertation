\chapter{Acknowledgments} %I use an asterisk because I don't want my acknowledgements in the table of contents. I use \chapter to make sure that the acknowledgements go on the correct side of the page when you print out the dissertation.

A thesis is not produced in a vacuum.
Countless people have contributed directly and indirectly over the years, and I can only highlight some of them here.

Mick and Janice, my parents and first teachers, provided no end of emotional, developmental, and financial support.
They never ceased to be available for me and I would not be here without them.
My siblings Rachel, Josh, and David each helped me grow in their own ways during my childhood and to this day.

I am especially grateful to my advisor Brian Cole for his continual mentorship during my graduate career.
He has taught me a lot about the experimental process and I can only hope to one day have a fraction of his scientific instincts.
His energy, generosity, and flexibility have been helpful far beyond the level of obligation.
Besides all this, he has excellent taste in beer.

I would like to thank many others in the Columbia heavy ion group as well.
Bill Zajc was an incredible resource particularly when I was a fresh new grad student trying to wrap my head around the kind of measurements in this thesis.
Aaron Angerami in particular was extremely generous with his time, going well out of his way to train new students and always being available for any sort of question.
The other postdocs in the group, Martin Ryb\'a\v{r}, Sarah Campbell, and especially Soumya Mohapatra have taught me a lot over the years as well.
I also thank my co-temporal graduate students Tingting Wang, Laura Havener, Yun Tian, and Xiao Tu, who have answered many questions and have always been a pleasure to work with.
I appreciate my academic sister Laura especially for being a wonderfully supportive friend through our time at Columbia, CERN, and various conferences, as she always provided encouragement when it was most needed.

I am grateful for the many other friends from my cohort at Columbia, with whom I've worked on problem sets, qualifying exams, and other technical questions, including Russell, Matt, Angelo, Ryne, and my long-time roommate Zach.
More importantly, they have kept my graduate school life enjoyable outside of physics.

Sathya Guruswamy, my undergraduate mentor, is an exceptional teacher not just in the classroom but also out of it.
I appreciate David Stuart for hiring me into his lab as an undergraduate where I got my first exposure to high energy physics.
My undergrad labmates Brad Axen, Devon Hollowood, and Leah Flink have taught me more than I can remember about physics, technology, and life and I am grateful we have been able to remain good friends despite scattering across the country.

Several other members of the ATLAS heavy ion group have generously lent me their assistance over the years, both with regards to analyses and during the November ion runs.
In particular Peter Steinberg, Dominik Derendarz, and Tomasz Bold have been especially helpful in their participation on the editorial boards for the papers.
In addition I appreciate the many various members of the ATLAS inner detector performance group for always being generous, helpful, and friendly.

Finally I am eternally grateful to my partner, Kate, for her constant emotional support and encouragement, especially during the stressful period at the end of my graduate career.
Personality traits that can be beneficial for science are often not the easiest to deal with on a personal level, and she has been patient, accepting, and understanding of that.
For that I am more than lucky to have her.
