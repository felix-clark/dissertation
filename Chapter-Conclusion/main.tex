\chapter{Conclusion}
\label{ch:conclusion}

The results in this dissertation comprise many detailed measurements of the two-pion correlation functions in \pPb collisions at \pPbenergy with the ATLAS detector at the \lhc.
The size and shape of the particle source are presented as a function of event centrality, transverse momentum, rapidity, and azimuthal angle from the second-order event plane.
These include the Lorentz invariant HBT radius \Rinv as well as the main 3D radii \Rout, \Rside, and \Rlong.
Cross-terms \Ros (coupling ``out'' and ``side'' components) and \Rol (coupling ``out'' and ``long'') are included in the azimuthally- and rapidity-dependent analyses, respectively.

The procedures developed in these results include some major technological improvements in the analysis procedure.
A data-driven technique was developed for constraining the contribution of jet fragmentation to the correlation function, which is a dominant systematic in small systems, particularly in \pp and peripheral \pA collisions.
This approach reduces the number of assumptions necessary for describing this background.
A sampling method is also developed to correct the azimuthally-dependent correlation functions for the event plane resolution, addresses an oversight made in the literature up to this point.

The HBT radii, representing the size of the source's region of homogeneity along the outwards (along \kt), sideways (other transverse), and longitudinal axes, are significantly larger in central collisions.
In central collisions, the radii exhibit a strong decrease with rising \kt, which is indicative of collective expansion.
This trend is significantly diminished in peripheral collisions, where within uncertainties there is no significant slope of the radii with respect to \kt.
Hydrodynamics may therefore be an appropriate description of central \pPb collisions, but not succeed in peripheral ones.

Within a \kt interval, the source radii increase monotonically from peripheral to central collisions.
At lower \kt the slope of this increase is larger, and the radii are proportional to \(\avgdNdeta^{1/3}\).
The radii are evaluated as a function of the rapidity-dependent multiplicity \dNdy, and each of them fall on a single curve, so they appear to depend only on the local density.

A nonzero cross-term \Rol coupling the ``out'' and ``long'' components is observed at low \kt in the forward direction of central events.
This demonstrates a breaking of the boost-invariance of the source function on the proton-going side.
In hydrodynamic models this indicates both longitudinal and transverse expansion, and indeed hydrodynamic predictions reproduce the rapidity dependence of \Rol in central (0--1\%) collisions.
This gives another indication of hydrodynamics that is only significant in central events.

The transverse HBT area $\detRt = \Rout \Rside - \Ros^2$ is proportional to the local multiplicity \dNdy at low \kt, suggesting a constant areal density.
At higher \kt ($\gtrsim 0.5 \GeV$), instead the volume element \detR is linear in the multiplicity.
The freeze-out volume increases steadily with \avgNpart.
With the standard Glauber geometry, the ratio of \detR to \Npart rises rapidly around $\avgNpart \gtrsim 12$, but in a \ac{GGCF} model the increase is much more modest.
A strict linear scaling of \detR with \avgNpart is not necessarily expected, but extreme deviations like that shown with the Glauber model are difficult to explain.
Previous results have supported the \ac{GGCF} models over standard Glauber, and these results provide some additional evidence for this view.
Even with allowing for fluctuations in the proton size, the expansion factor (i.e. ratio of final volume to initial size) still appears to be greater in central events, possibly indicating a turn-on of hydrodynamics.

The ratio of \Rout to \Rside is less than unity for all centrality and kinematic selections.
The value of $\Rout/\Rside$ is interpreted as indicating the lifetime of the source, since in hydrodynamic models the lifetime contributes to \Rout but not to \Rside.
It decreases with rising \kt, which is consistent with the description that higher-momentum particles freeze out from the source at earlier times.
The small value of $\Rout/\Rside$ is consistent with an explosive expansion of the source.
There is little centrality dependence, although it is slightly higher in very central events, and there is no significant rapidity dependence of $\Rout/\Rside$.

The modulation of the HBT radii with respect to the second-order event plane in very central \pPb events with $\Nch \geq 150$ is also measured.
In events with a large flow $|\qt|$, the second-order Fourier components of the HBT radii can be extracted with good precision.
In events with low anisotropy, the poor event plane resolution precludes distinguishing the modulation from zero.
A similar dependence of the radii on \tdpk relative to the second-order event plane is observed as in \AA collisions.
Transverse radii are suppressed in-plane and enhanced out-of-plane, which is the same orientation required for the initial conditions in hydrodynamic models with a short lifetime of the medium.
The \kt-dependence of the Fourier components of the radii is also consistent with the predictions of hydrodynamics, and it shows that the ellipticity of the expanding source decreases during the duration of its evolution.
The longitudinal ratio \Rlong modulation indicates that the medium has greater longitudinal expansion along the event plane.
These results support the interpretation that short-lived hydrodynamic evolution is a source of the flow-like azimuthal multiplicity distributions in central \pPb events, and present a significant challenge to competing descriptions that do not directly link initial geometry to final-state momentum distributions.

In summary, the results presented in this dissertation provide detailed measurements of the \pPb source density as a function of all the most significant event-level and kinematic variables.
The highly-differential measurements include the first rapidity- and azimuthally-dependent femtoscopic results in \pA collisions.
Theoretical computations have already shown some success in describing central \pPb events, and the opportunity to post-dict the Fourier components of the azimuthal HBT radii is available.
The evidence for hydrodynamic behavior is compelling in central \pPb; however, it remains incumbent upon the field to determine more precisely how and where the onset of hydrodynamics occurs.

Recent theoretical work has provided some explanation for the unreasonable success of hydrodynamics, even in systems that are not expected to last long enough to reach full thermal equilibrium.
Experiments should continue to work to establish a precise quantitative description of the domain of applicability of hydrodynamics.
This is challenging, since the turn-on is not likely to occur suddenly in any of the relevant physical variables like \Npart, multiplicity, and transverse momentum.
It now seems clear that central \pPb collisions evolve through hydrodynamics, but measurements of collective and hydrodynamic behavior need to be refined in peripheral \pPb and \pp collisions, where the systematic effects are much more significant due to lower \Nch, poorer event plane resolution, and the increased relative effects of jets.

\todo{
evaluate ``turn-on'' of hydro in \pPb / \pp collisions.
apply improvements in jet description to determine \kt dependence of HBT radii in \pp, as function of multiplicity. (check CMS result from sandra padula at WPCF: not published yet but record at https://cds.cern.ch/record/2318575)
push azimuthal to more peripheral (probably difficult w/out many more statistics)
jet quenching (dijet asymmetry, gamma-jet) in ultra-central \pPb?
}
