\chapter{Conclusion}
\label{ch:conclusion}

The results in this dissertation comprise many detailed measurements of the two-pion correlation functions in \pPb collisions at \pPbenergy with the ATLAS detector at the \lhc.
The size and shape of the particle source are presented as a function of event centrality, transverse momentum, rapidity, and azimuthal angle from the second-order event plane.
These include the Lorentz invariant HBT radius \Rinv as well as the main 3D radii \Rout, \Rside, and \Rlong.
Cross-terms \Ros (coupling ``out'' and ``side'' components) and \Rol (coupling ``out'' and ``long'') are included in the azimuthally- and rapidity-dependent analyses, respectively.

The procedures developed in these results include some major technological improvements in the analysis procedure.
A data-driven technique was developed for constraining the contribution of jet fragmentation to the correlation function, which is a dominant systematic in small systems, particularly in \pp and peripheral \pA collisions.
This approach reduces the number of assumptions necessary for describing this background.
A sampling method is also developed to correct the azimuthally-dependent correlation functions for the event plane resolution, addresses an oversight made in the literature up to this point.

The HBT radii, representing the size of the source's region of homogeneity along the outwards (along \kt), sideways (other transverse), and longitudinal axes, are significantly larger in central collisions.
In central collisions, the radii exhibit a strong decrease with rising \kt, which is indicative of collective expansion.
This trend is significantly diminished in peripheral collisions, where within uncertainties there is no significant slope of the radii with respect to \kt.
Hydrodynamics may therefore be an appropriate description of central \pPb collisions, but not succeed in peripheral ones.

Within a \kt interval, the source radii increase monotonically from peripheral to central collisions.
At lower \kt the slope of this increase is larger, and the radii are proportional to \(\avgdNdeta^{1/3}\).
The radii are evaluated as a function of the rapidity-dependent multiplicity \dNdy, and each of them fall on a single curve, so they appear to depend only on the local density.

A nonzero cross-term \Rol coupling the ``out'' and ``long'' components is observed at low \kt in the forward direction of central events.
This demonstrates a breaking of the boost-invariance of the source function on the proton-going side.
In hydrodynamic models this indicates both longitudinal and transverse expansion, and indeed hydrodynamic predictions reproduce the rapidity dependence of \Rol in central (0--1\%) collisions.
This gives another indicator of hydrodynamics that is only significant in central events.



``for the future'' ideas:\\
evaluate ``turn-on'' of hydro in \pPb / \pp collisions.
apply improvements in jet description to determine \kt dependence of HBT radii in \pp, as function of multiplicity. (check CMS result from sandra padula at WPCF: not published yet but record at https://cds.cern.ch/record/2318575)
push azimuthal to more peripheral (probably difficult w/out many more statistics)
jet quenching (dijet asymmetry, gamma-jet) in ultra-central \pPb?
